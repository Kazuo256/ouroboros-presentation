\documentclass[brazil]{beamer}
\usepackage{beamerthemesplit}
\usepackage[brazilian]{babel}
\usepackage[utf8]{inputenc}
\usepackage{color}
\usepackage{xcolor}
\usepackage{amssymb}
\usepackage{amsmath}
\usepackage{fancybox}
\usepackage{ulem}
\usepackage{listings}
\usepackage{upquote}
\usetheme{JuanLesPins}
%\usetheme{Warsaw}
%\usetheme{CambridgeUS}
%\usetheme{Malmoe}


%\newcommand{\lyxline}[1][1pt]{%
%  \par\noindent%
%  \rule[.5ex]{\linewidth}{#1}\par}


\title{
  Projeto Ouroboros
}
\subtitle{
  Um sistema de integração automatizada entre C++ e
  linguagens de script
}
\author{Wilson Kazuo Mizutani e Fernando Omar Aluani}

\begin{document}

% ---------------------------------------------------------------------------- %
% Opções de listing usados para o código fonte
% Ref: http://en.wikibooks.org/wiki/LaTeX/Packages/Listings
\lstset{ %
  language=C++,                     % choose the language of the code
  basicstyle=\ttfamily\scriptsize,  % the size of the fonts that are used for the code
  numbers=left,                     % where to put the line-numbers
  numberstyle=\footnotesize,        % the size of the fonts that are used for the line-numbers
  stepnumber=1,                     % the step between two line-numbers. If it's 1 each line will be numbered
  numbersep=5pt,                    % how far the line-numbers are from the code
  showspaces=false,                 % show spaces adding particular underscores
  showstringspaces=false,           % underline spaces within strings
  showtabs=false,                   % show tabs within strings adding particular underscores
  %frame=single,                     % adds a frame around the code
  %framerule=0.6pt,
  tabsize=2,                        % sets default tabsize to 2 spaces
  captionpos=b,                     % sets the caption-position to bottom
  breaklines=true,                  % sets automatic line breaking
  breakatwhitespace=false,          % sets if automatic breaks should only happen at whitespace
  escapeinside={\%*}{*)},           % if you want to add a comment within your code
  %backgroundcolor=\color[rgb]{1.0,1.0,1.0}, % choose the background color.
  %rulecolor=\color[rgb]{0.8,0.8,0.8},
  basicstyle=\ttfamily\scriptsize,
  keywordstyle=\color{blue}\bfseries,
  keywordstyle=[2]\color[rgb]{.4,0,.4}\bfseries,
  commentstyle=\color[rgb]{0,.6,0},
  stringstyle=\color{red},
  showstringspaces=false,
  upquote=true,
  extendedchars=true,
  %xleftmargin=10pt,
  %xrightmargin=10pt,
  %framexleftmargin=10pt,
  %framexrightmargin=10pt,
  morekeywords={[2]include,ifdef,define,ifndef,endif,}
}

\frame{\titlepage}

\frame{\tableofcontents}

%-------------------------------------
\section{Introdução e breve histórico}
%-------------------------------------
\frame{
  \begin{center}
  \LARGE 1. Introdução e breve histórico
  \end{center}
}
%-------------------------------------
\frame{
  \frametitle{Histórico}
  \begin{itemize}
    \item Projeto da PUC-Rio (Tecgraf e LabLua)
    \pause
    \item Site oficial: www.lua.org
    \pause
    \item Últimas versões:
    \pause
    \begin{itemize}
      \item 5.0 (obsoleto)
      \pause
      \item 5.1 (atual)
      \pause
      \item 5.2 (muito hipster ainda)
    \end{itemize}
  \end{itemize}
}
%-------------------------------------
\frame{
  \frametitle{Principais características}
  \begin{itemize}
    \pause
    \item Simplicidade
    \pause
    \item Eficiência
    \pause
    \item Versatilidade
    \pause
    \item Leveza
  \end{itemize}
}
%-------------------------------------
\begin{frame}[fragile]
  \frametitle{Exercício 1.1: Hello World!!!}
  \pause
  \begin{columns}
    \column[t]{.5\textwidth}
      \begin{block}{helloworld.lua}
        \begin{lstlisting}
// My hello world Lua program:
print("Hello World!");
        \end{lstlisting}
      \end{block}
    \pause
    \column[t]{.5\textwidth}
      \begin{block}{Saída}
        \begin{verbatim}
Hello World!  \end{verbatim}
      \end{block}
  \end{columns}
\end{frame}
%-------------------------------------
\section{Expressões básicas}
%-------------------------------------
\frame{
  \begin{center}
  \LARGE 2. Expressões básicas
  \end{center}
}
%-------------------------------------
\subsection{2.1. Tipos escalares.}
%-------------------------------------
\frame{
  \begin{center}
  \Large 2.1. Tipos escalares
  \end{center}
}
%-------------------------------------
\frame{
  \frametitle{Tipagem Dinâmica}
  \begin{itemize}
    \pause
    \item Comum em linguagens de script.
    \pause
    \item O tipo de uma variável é definido pelo valor atual dela.
    \pause
    \item Portanto, o tipo de uma variável pode mudar ao longo do
          programa.
  \end{itemize}
}
%-------------------------------------
\subsection{2.2. Operações básicas}
%-------------------------------------
\frame{
  \begin{center}
  \Large 2.2. Operações básicas
  \end{center}
}
%-------------------------------------
%% TODO: exercício
\frame{
  \frametitle{Exercício 2.1}
}
%-------------------------------------
\section{Expressões de controle}
%-------------------------------------
\frame{
  \begin{center}
  \LARGE 3. Expressões de controle
  \end{center}
}
%-------------------------------------
\subsection{3.1. Estruturas}
%-------------------------------------
\frame{
  \begin{center}
  \Large 3.1. Estruturas
  \end{center}
}
%-------------------------------------
%% TODO: exercício
\frame{
  \frametitle{Exercício 3.1}
}
%-------------------------------------
%% TODO: exercício
%% sugestão: fazer um mesmo loop usando while, repeat e for.
\frame{
  \frametitle{Exercício 3.2}
}
%-------------------------------------
\subsection{3.2. Escopo.}
%-------------------------------------
\frame{
  \begin{center}
  \Large 3.2. Escopo
  \end{center}
}
%-------------------------------------
\frame{
  \frametitle{Blocos}
  \begin{itemize}
    \pause
    \item Um script lua apresenta uma hierarquia de blocos.
    \pause
    \item Um bloco é uma sequência de comandos, como os
          exemplos que vimos até agora.
    \pause
    \item Um bloco pode ter outros blocos dentro de si,
          através de:
    \begin{itemize}
      \pause
      \item estruturas de controle (if, while, for, etc);
      \pause
      \item definições de funções; ou
      \pause
      \item uso explícito de "do..end".
    \end{itemize}
  \end{itemize}
}
%-------------------------------------
\frame{
  \frametitle{Visibilidade}
  \begin{itemize}
    \pause
    \item Variáveis globais são visíveis a partir de
          qualquer bloco, contanto que não esteja
          obscurecida por alguma variável local visível
          de mesmo nome.
    \pause
    \item Variáveis locais são visíveis no bloco em que
          foram declaradas e nos blocos aninhados nele,
          contanto que sejam referenciadas após a linha em
          que foram definidas.
    \pause
    \item Se um bloco tenta acessar uma variável que não
          está visível de jeito nenhum para ele, ele
          receberá um 'nil'.
  \end{itemize}
}
%-------------------------------------
%% TODO: exercício "o i do for é global ou local?"
\frame{
  \frametitle{Exercício 3.3}
}
%-------------------------------------
\section{Funções}
%-------------------------------------
\frame{
  \begin{center}
  \LARGE 4. Funções
  \end{center}
}
%-------------------------------------
\subsection{4.1. Criando e usando funções}
%-------------------------------------
\frame{
  \begin{center}
  \Large 4.1. Criando e usando funções
  \end{center}
}
%-------------------------------------
%% TODO exercício
\frame{
  \frametitle{Exercício 4.1}
}
%-------------------------------------
%% TODO exercício
\frame{
  \frametitle{Exercício 4.2}
}
%-------------------------------------
\subsection{4.2. Upvalues}
%-------------------------------------
\frame{
  \begin{center}
  \Large 4.2. Upvalues
  \end{center}
}
%-------------------------------------
%% TODO exercício
%% - a variável "upvalue" era necessária?
\frame{
  \frametitle{Exercício 4.3}
}
%-------------------------------------
\section{Tabelas}
%-------------------------------------
\frame{
  \begin{center}
  \LARGE 5. Tabelas
  \end{center}
}
%-------------------------------------
\subsection{5.1. Criando e manipulando tabelas}
%-------------------------------------
\frame{
  \begin{center}
  \Large 5.1. Criando e manipulando tabelas
  \end{center}
}
%-------------------------------------
%% TODO: exercício
\frame{
  \frametitle{Exercício 5.1}
}
%-------------------------------------
%% TODO: exercício
\frame{
  \frametitle{Exercício 5.2}
}
%% - sugestão: construir "tabuada"
%-------------------------------------
%% TODO: exercício
\frame{
  \frametitle{Exercício 5.3}
}
%% - desafio: qual a diferença entre
%%   usar pairs e usar foreach?
%-------------------------------------
%% TODO exercício
\frame{
  \frametitle{Exercício 5.4}
}
%-------------------------------------
\begin{frame}[fragile]
  \frametitle{Possibilidades e restrições}
  \begin{itemize}
    \pause
    \item Praticamente QUALQUER COISA pode ser inserido em uma
          tabela de Lua, seja como chave ou como valor.
    \pause
    \item Isso significa que tabelas podem ter outras tabelas e
          até mesmo funçoes como chaves e valores!
    \pause
    \item A ÚNICA EXCEÇÃO é o \verb$nil$:
    \begin{itemize}
      \pause
      \item Chaves NUNCA podem ser \verb$nil$ (isso causa um erro)
      \pause
      \item Campos com valores \verb$nil$ são removidos da tabela.
    \end{itemize}
  \end{itemize}
\end{frame}
%-------------------------------------
%% TODO exercício
\frame{
  \frametitle{Exercício 5.5}
}
%-------------------------------------
\subsection{5.2. Atalhos}
%-------------------------------------
\frame{
  \begin{center}
  \Large 5.2. Atalhos
  \end{center}
}
%-------------------------------------
%% TODO exercício
\frame{
  \frametitle{Exercício 5.6}
}
%-------------------------------------
\section{Bibliotecas básicas}
%-------------------------------------
\frame{
  \begin{center}
    \LARGE 6. Bibliotecas básicas
  \end{center}
}
%-------------------------------------
\begin{frame}
  \vspace{15pt}
  Em inglês: \url{http://www.lua.org/manual/5.1/}

  Em português: \url{http://www.lua.org/manual/5.1/pt/}
\end{frame}
%-------------------------------------
%% TODO exercício
\frame{
  \frametitle{Exercícios 6.*}
}
%-------------------------------------
\section{Unlimited slide works}
%-------------------------------------
\frame{
  \begin{center}
    \LARGE ENFIM...
  \end{center}
}
%-------------------------------------
\begin{frame}
  \frametitle{Tópicos avançados que não aparecem nos slides}
  \begin{itemize}
    \item Tipos thread e userdata.
    \item Bibliotecas avançadas: module. debug e thread.
    \item Ambientes de funções.
    \item Metatabelas.
    \item API C.
  \end{itemize}
\end{frame}
%-------------------------------------
\begin{frame}
  \begin{center}
    \LARGE Obrigado!
  \end{center}
  \begin{itemize}
    \item Visitem www.uspgamedev.org
    \item Dúvidas: kazuo@uspgamedev.org
    \item Se seu interesse for desenvolvimento de games,
          vale MUITO a pena dar uma conferida na engine
          LÖVE2D.
    \begin{itemize}
      \item Site oficial: www.love2d.org
      \item Jogo feito com LÖVE2D: http://stabyourself.net/mari0
    \end{itemize}
  \end{itemize}
\end{frame}
%-------------------------------------
\end{document}

